\documentclass[10pt,a4paper,oneside,parskip=half]{scrartcl}
\usepackage[utf8]{inputenc}
\usepackage[ngerman]{babel}
\usepackage[T1]{fontenc}
\usepackage[margin=2.5cm]{geometry}
\usepackage{lmodern}
\usepackage{graphicx}
\usepackage{tabularx}
\usepackage{xfrac}
\usepackage[juratotoc]{scrjura}
\makeatletter
\providecommand*{\toclevel@cpar}{0}
\makeatother
\usepackage{lastpage}
\usepackage{scrpage2}
\usepackage{hyperref}

\title{Satzung}
\date{}
\author{}

\pagestyle{scrheadings}
\renewcommand*{\titlepagestyle}{scrheadings}

\begin{document}
\cfoot{Seite \thepage\ von \pageref{LastPage}}
\maketitle

\begin{contract}

\addsec{Vorwort}
Die Oldenburger Jugendverbände, Jugendgemeinschaften, Initiativen der Jugend und Jugendvereinigungen haben sich zum Stadtjugendring zusammengeschlossen, um in freiwilliger Zusammenarbeit ihre gemeinsamen Interessen zu fördern und die Belange der Jugend zu vertreten. Die Unabhängigkeit der Jugendverbände, Jugendgemeinschaften, Initiativen der Jugend und Jugendvereinigungen wird dadurch nicht beeinträchtigt.

\Clause{title=Name und Sitz}
Der Stadtjugendring Oldenburg ist ein freiwilliger Zusammenschluss von Jugendverbänden, Jugendgemeinschaften, Initiativen der Jugend und Jugendvereinigungen und jugendpflegerisch tätigen Vereinen im Stadtgebiet Oldenburg.
Er trägt den Namen Stadtjugendring Oldenburg~e.\,V. Sein Sitz ist im Stadtgebiet Oldenburg. Der Verein ist im Vereinsregister Oldenburg eingetragen.

\Clause{title=Aufgaben und Ziele}
\label{aufgaben}
Aufgaben des Stadtjugendringes (im Folgenden SJR genannt) sind:
\begin{enumerate}
\item das gegenseitige Verständnis und die Bereitschaft zur Zusammenarbeit in der jungen Generation durch ständigen Erfahrungsaustausch und gegenseitige Unterstützung zu fördern,
\item die Interessen von Jugendlichen und Kindern, ihrer Gruppen, Zusammenschlüsse und Jugendverbände in der Öffentlichkeit und gegenüber Parlamenten und Behörden durch eine qualifizierte Mitbestimmung zu vertreten (z.\,B. Jugendhilfeausschuss, Jugendamt, Jugendpflege),
\item die Mitglieder zu gemeinsamen Aktionen und Veranstaltungen der außerschulischen Bildung anzuregen, diese gemeinsam zu planen und sie bei der Durchführung mit personellen und finanziellen Ressourcen zu unterstützen,
\item gemeinsame Einrichtungen (z.\,B. Jugendzentren, Jugendhäuser) zu initiieren,
\item Stellungnahmen, Informationsschriften, Arbeitsmaterial und Publikationen zu jugend- und kinderpolitischen Themen, parteipolitisch unbeeinflusst, herauszugeben,
\item die internationale Kinder- und Jugendzusammenarbeit, Begegnungen und Studienfahrten zum Kennenlernen gesellschaftlicher Probleme anderer Länder als Beitrag der Völkerverständigung anzuregen und zu fördern,
\item der Anwendung von Gewalt als Mittel der gesellschaftlichen und politischen Auseinandersetzung, verfassungs- und gesellschaftsfeindlichen, insbesondere nationalistischen und rassistischen Tendenzen mit aller Kraft entgegenzuwirken.
\end{enumerate}

\Clause{title=Gemeinnützigkeit}
Der SJR verfolgt ausschließlich und unmittelbar gemeinnützige Zwecke im Sinne des Abschnittes „Steuerbegünstigte Zwecke“ der Abgabenordnung, insbesondere durch die Förderung der Kinder- und Jugendarbeit im Sinne des §~11 Abs.~3 und 12~KJHG. Einnahmen dürfen nur für die satzungsgemäßen Zwecke verwendet werden. Die Mitglieder können Zuwendungen aus Mitteln des Stadtjugendringes Oldenburg zur Umsetzung der Aufgaben und Ziele erhalten. Der Verein ist selbstlos tätig. Es darf keine Person durch Ausgaben, die dem Zweck der Körperschaft fremd sind, oder durch unverhältnismäßig hohe Vergünstigungen begünstigt werden.

Es dürfen Personen durch Ausgaben begünstigt werden, die dem Zweck der Körperschaft nützlich sind oder als Aufwandsentschädigung z.\,B. Webmastertätigkeit, Verwaltungsarbeit, Telefon- und Fahrkosten etc. bezeichnet werden können.

Der Vorstand kann für seine Tätigkeit eine Vergütung im Rahmen des §~3 Nr.~26a~EStG erhalten. Über die Höhe dieser Vergütung entscheidet die Vollversammlung.

Der SJR kann haupt- und nebenberufliche Mitarbeiter/innen zur Erfüllung seiner Aufgaben beschäftigen. Diesen gegenüber wird der SJR durch den Vorstand vertreten. Die Aufsicht wird durch die oder den VorsitzendeN oder einem/einer stellvertretenden Vorsitzenden oder durch eine/einen von der oder dem Vorsitzenden benannte VertreterIn wahrgenommen.

\Clause{title=Mitgliedschaft}
Mitglied des SJR können alle im Stadtgebiet tätigen Jugendverbände, Jugendgemeinschaften, Initiativen der Jugend und Jugendvereinigungen, jugendpflegerisch tätige Vereine sowie Organisationen und Zusammenschlüsse Jugendlicher werden, die in Form von rechtsfähigen oder nicht rechtsfähigen Vereinen organisiert sind. Diese werden im Folgenden Mitglieder genannt.

Eine Mitgliedschaft verpflichtet zur Mitarbeit.\label{mitgliedschaft_mitarbeit}

Die Voraussetzung für die Mitgliedschaft im Stadtjugendring Oldenburg ist:
\begin{enumerate}
\item Anerkennung der Bundesrepublik Deutschland mit den im Grundgesetz verankerten Grundrechten, sowohl in der Zielsetzung als auch in der praktischen Arbeit.
\item Für Jugendverbände, Jugendgemeinschaften, Initiativen der Jugend und Jugendvereinigungen, die einem Erwachsenenverband angehören, dass sie eine selbständige und selbst bestimmte Arbeit leisten.
\item Anerkennung dieser Satzung.
\end{enumerate}

Die Mitgliedschaft ist schriftlich unter Beigabe einer Satzung oder Ordnung und der Darstellung der tatsächlich stattfindenden Arbeit beim Vorstand zu beantragen. Der Vorstand prüft die Formalitäten und reicht den Antrag an die Vollversammlung weiter, welche über die Aufnahme mit einfacher Mehrheit beschließt.

Wird ein Antrag auf Aufnahme abgelehnt, kann nach einer Frist von sechs Monaten ein neuer Antrag gestellt werden.

Ein dauerhafter Zugang zu Ressourcen (Vereins- und Gemeinschaftsräume, Mittel etc.) des SJR sollte grundsätzlich nur den Mitgliedern gewährt werden. Bei Erlöschen der Mitgliedschaft (\ref{mitgliedschaftsende}) wird der Zugang zu den Ressourcen mit sofortiger Wirkung verwehrt.

Ein Mitgliedsbeitrag wird nicht erhoben.\label{mitgliedschaft_beitrag}

Außerordentliche Mitglieder können alle Vereine, Organisationen und Gemeinschaften werden, die sich, ohne auf dem Gebiet der Jugendarbeit tätig zu sein, mit der Interessenvertretung für ihre Kinder und jugendlichen Mitglieder befassen. \refPar{mitgliedschaft_mitarbeit} bis \refParN[arabic]{mitgliedschaft_beitrag} gelten entsprechend. Der Hauptausschuss kann über eine außerordentliche Mitgliedschaft entscheiden. Außerordentliche Mitglieder nehmen beratend an den Versammlungen und Ausschüssen teil.

\Clause{title=Organe}
Der SJR hat folgende Organe:
\begin{enumerate}
\item Die Vollversammlung
\item Den Vorstand
\item Den Hauptausschuss
\end{enumerate}

\Clause{title=Vollversammlung}
Der Vollversammlung (im Folgenden VV genannt) gehören je ein/e stimmberechtigte/r Vertreter/in der Mitglieder an. Die Vorstandsmitglieder gehören der VV an. Sie haben jedoch kein Stimmrecht, soweit sie nicht Vertreter im Sinne von Satz~1 sind. Die VV kann beschließen, weitere Personen ohne Stimmrecht generell oder im Einzelfall zu den Sitzungen hinzuzuziehen. Ferner ist der/die Stadtjugendpfleger/in der Stadt Oldenburg beratendes Mitglied der VV. Die Mitglieder regeln ihre Vertretung selber. Sie teilen dem Vorstand mit, durch wen sie sich vertreten lassen.

Die VV ist öffentlich. Auf Antrag kann die VV, mit einfacher Mehrheit, die Öffentlichkeit für einzelne Punkte der Tagesordnung ausschließen.

Die VV tritt mindestens einmal im Jahr zusammen. Der Vorstand beruft die VV unter Bekanntgabe der Tagesordnung mindestens einen Monat vor Sitzungstermin schriftlich ein. Es gilt der Poststempel.

Wird von zwei Fünftel der Mitglieder die Einberufung der VV gefordert, so muss der Vorstand sie innerhalb von 14~Tagen schriftlich einberufen.

Die VV ist das oberste Beschlussorgan des SJR. Sie legt Richtlinien der Arbeit fest, entlastet den Vorstand und wählt den Vorstand nach §~26~BGB, evtl. den erweiterten Vorstand und 2~Revisor/innen. Die Vollversammlung kann sich im Rahmen dieser Satzung eine Geschäftsordnung geben.

Die VV ist beschlussfähig, wenn satzungsgemäß eingeladen wurde und mindestens die Hälfte der stimmberechtigten Mitglieder anwesend sind. Sollte eine satzungsgemäß eingeladene VV nicht beschlussfähig sein, so kann der Vorstand fristgemäß eine zweite VV schriftlich einberufen. Diese ist ohne Rücksicht auf die Anzahl der anwesenden stimmberechtigten Mitglieder beschlussfähig; hierauf ist in der Einladung besonders hinzuweisen.

Beschlüsse und Planung erfolgen mit absoluter Mehrheit der anwesenden Stimmberechtigten, solange diese Satzung nichts anderes vorsieht. Bei Satzungsänderungen ist eine Zweidrittelmehrheit erforderlich.

Wahlen werden offen durchgeführt. Sie müssen auf Antrag einer stimmberechtigten Person geheim durchgeführt werden.

Für besondere Aktivitäten und Aufgaben kann die VV thematische Projektgruppen einrichten. Sie kann diese Aufgabe auch dem Vorstand übertragen. Die Teilnehmer/innen dieser Projektgruppen nehmen an den Sitzungen der Vollversammlung beratend teil.

Der SJR gibt sich eine Kassenordnung, die durch die VV bestätigt werden muss.

Über die Vollversammlung wird Protokoll geführt. Das Protokoll ist von der Sitzungsleitung und der Protokoll führenden Person zu unterschreiben.

Anträge zur VV können vom Vorstand und den Mitgliedern eingebracht werden. Die Anträge der Mitglieder müssen zwei Wochen vor der VV schriftlich beim Vorstand eingereicht werden. Tagesordnungspunkte, die aus der VV heraus eingebracht werden, bedürfen der Zustimmung von einem Drittel der anwesenden Stimmberechtigten, wenn sie behandelt werden sollen.

\Clause{title=Der Vorstand}
Der Vorstand im Sinne des §~26~BGB setzt sich zusammen aus der/dem Vorsitzenden, einen bis drei stellvertretenden Vorsitzenden und der/dem Kassenwart/in. Ein Maximum ist anzustreben. Der Vorstand führt die Geschäfte des Vereins im Sinne des §~26~BGB. Hierzu sind jeweils zwei Mitglieder des Vorstandes gemeinsam berechtigt.

Der Vorstand wird von der Vollversammlung für eine zweijährige Amtsperiode gewählt. Bei Ausscheiden von Mitgliedern aus dem Vorstand kann auf der nächsten Vollversammlung nachgewählt werden. Der Vorstand bleibt bis zur Neuwahl eines Vorstandes im Amt.

Die Vorstandsmitglieder werden einzeln in getrennten Wahlgängen gewählt. Für die Wahlen zum Vorstand bedarf es einer absoluten Mehrheit der anwesenden stimmberechtigten Mitglieder. Kommt diese im ersten Wahlgang nicht zustande, findet ein zweiter Wahlgang statt, bei dem die einfache Mehrheit der anwesenden stimmberechtigten Mitglieder entscheidet.

Alle Vorstandsmitglieder sollten aus unterschiedlichen Mitgliedern gewählt werden. Maximal dürfen zwei Vorstandsmitglieder aus einem Mitglied kommen.

Der Vorstand beruft die Vollversammlungen ein, ist für die Tagesordnung verantwortlich, bearbeitet die laufenden Aufgaben und führt die Geschäfte des Stadtjugendringes nach Maßgabe der Vollversammlung.

Der Vorstand gibt sich eine Geschäftsordnung. Im Rahmen dieser Geschäftsordnung wird die Aufgabenteilung festgelegt. Die Vorstandssitzungen sind für die Mitgliedsverbände öffentlich. Auf Verlangen eines Vorstandsmitglieds kann die Öffentlichkeit ausgeschlossen werden. Über die Ergebnisse der Vorstandssitzung wird Protokoll geführt.

Der Vorstand ist gehalten, bei Verträgen, die den Verein länger als ein Jahr binden, Kreditaufnahmen aller Art, außerplanmäßigen Aufwendungen, die ohne entsprechende zusätzliche Einnahmen mehr als 10\% des Jahresetats ausmachen, sich durch Fachleute beraten zu lassen. Das Ergebnis solcher Beratungen ist schriftlich festzuhalten.

Der Vorstand ist beschlussfähig, wenn mindestens die Hälfte seiner Mitglieder erschienen sind. Er fasst seine Beschlüsse mit Stimmenmehrheit. Die Beschlussunfähigkeit ist auf Antrag festzustellen. Alle Vorstandsmitglieder sind in gleicher Weise stimmberechtigt.

Zur Führung der laufenden Geschäfte kann der Vorstand Mitarbeiter/innen einstellen oder Personen beauftragen. Diese nehmen beratend an den Sitzungen des Vorstandes teil. Die Einstellung oder Beauftragung muss von der nächstfolgenden VV bestätigt werden.

Jedes Vorstandsmitglied ist im Rahmen der gesetzlichen Vorschriften zum Ersatz des Schadens verpflichtet, den es vorsätzlich oder durch eine grob fahrlässige Sorgfaltspflichtverletzung dem Verein oder einem Dritten zufügt.

Der Vorstand kann postalisch, per E-Mail oder per Telefax Beschlüsse herbeiführen, wenn alle Vorstandsmitglieder dem zustimmen.

\Clause{title=Hauptausschuss}
Dem Hauptausschuss (im Folgenden HA genannt) gehören je ein/e stimmberechtigte/r Vertreter/in der Mitglieder an. Die Vorstandsmitglieder gehören dem HA an. Sie haben jedoch kein Stimmrecht, soweit sie nicht Vertreter im Sinne von Satz 1 sind. Der HA kann beschließen, weitere Personen ohne Stimmrecht generell oder im Einzelfall zu den Sitzungen hinzuzuziehen. Die Mitglieder regeln ihre Vertretung selber. Sie teilen dem/der Vorsitzenden mit, durch wen sie sich im Hauptausschuss vertreten lassen.

Der HA nimmt alle Aufgaben, insbesondere die aus \ref{aufgaben} war, soweit sie nicht ausdrücklich anderen Organen nach dieser Satzung vorbehalten sind. Er kann Arbeitsausschüsse einsetzen.

Der Vorstand lädt den HA etwa alle zwei Monate zu einer Sitzung unter Angabe der Tagesordnung mit einer Frist von zehn Tagen ein.

Der Hauptausschuss ist beschlussfähig, wenn mindestens die Hälfte der stimmberechtigten Mitglieder vertreten ist. Beschlüsse des Hauptausschusses werden mit der einfachen Mehrheit der anwesenden Stimmberechtigten gefasst.

\Clause{title=Revision}
Die Revisor/innen prüfen mindestens einmal jährlich die Ein- und Ausgaben des SJR auf die satzungsgemäße Verwendung und ordentliche Buchhaltung.

\Clause{title=Ende der Mitgliedschaft}
\label{mitgliedschaftsende}
Die Mitgliedschaft erlischt durch:
\begin{enumerate}
\item Austritt
\item Auflösung des Mitglieds
\item Ausschluss
\end{enumerate}

Der Austritt kann jederzeit schriftlich gegenüber dem Vorstand erklärt werden. Er wird mit dem Zugang des Schreibens wirksam.

Der Antrag auf Ausschluss kann vom Vorstand, dem HA oder von einem Drittel der Mitglieder an die VV gestellt werden, wenn ein Mitglied in erheblichem Umfang gegen die Satzung oder gegen satzungsgemäße Beschlüsse des SJR verstößt bzw. nicht mehr die Voraussetzungen für eine Mitgliedschaft erfüllt. Die VV muss über einen Ausschluss mit absoluter Mehrheit entscheiden.

\Clause{title=Auflösung}
Der Stadtjugendring Oldenburg~e.\,V. kann nur auf einer eigens hierzu einberufenen außerordentlichen Vollversammlung mit einer Mehrheit von \sfrac{3}{4} der anwesenden stimmberechtigten Mitglieder aufgelöst werden.
Im Falle der Auflösung des SJR oder bei Wegfall der steuerbegünstigten Zwecke fällt das Vermögen an die Stadtjugendpflege, die es ausschließlich für Zwecke der Förderung der Kinder- und Jugendverbandsarbeit zu verwenden hat.

\end{contract}
\vspace{1cm}
Stadt Oldenburg, 03. Dezember 2013
\end{document}